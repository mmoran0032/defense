\documentclass[10pt]{beamer}

\usetheme[progressbar=frametitle]{metropolis}
\usepackage{appendixnumberbeamer}

\usepackage{booktabs}
\usepackage[scale=2]{ccicons}

\usepackage{pgfplots}
\usepgfplotslibrary{dateplot}

\usepackage{xspace}
\newcommand{\themename}{\textbf{\textsc{metropolis}}\xspace}

\newcommand{\nuc}[2]{${}^{#1}\textrm{#2}$}
\newcommand{\mnuc}[2]{{}^{#1}\textrm{#2}}
\newcommand{\react}[4]{$#1(#2,#3)#4$}
\newcommand{\mreact}[4]{#1(#2,#3)#4}
\newcommand{\alpa}{\react{\mnuc{27}{Al}}{\textrm{p}}{\alpha}{\mnuc{24}{Mg}}}
\newcommand{\nag}{\react{\mnuc{14}{N}}{\alpha}{\gamma}{\mnuc{18}{F}}}
\newcommand{\squared}{${}^{2}$}

\title{
    Verification of Recoil Separator Properties Through Reaction Measurements
}
\subtitle{}
% \date{\today}
\date{November 9, 2018}
\author{Michael T Moran}
\institute{University of Notre Dame}
% \titlegraphic{\hfill\includegraphics[height=1.5cm]{logo.pdf}}

\begin{document}

\maketitle

\begin{frame}[fragile]{Table of contents}
  \setbeamertemplate{section in toc}[sections numbered]
  \tableofcontents[hideallsubsections]
\end{frame}

\section{Introduction}

\begin{frame}[fragile]{Prominent Reaction Processes}

    Stellar energy production depends on the type of star and the
    current lifecycle stage

    $pp$-Chains and CNO cycles dominate for stars similar to our sun
    \begin{itemize}
        \item Include radiative capture reactions: $({\rm p},\gamma)$
            and $(\alpha,\gamma)$
    \end{itemize}

    Can be grouped under ``Hydrogen burning'' processes

\end{frame}

\begin{frame}[fragile]{Hydrogen Burning}

    \begin{figure}
        \includegraphics[width=0.95\textwidth]%
            {figures/cno_nena_mgal.pdf}
    \end{figure}
\end{frame}

\begin{frame}[fragile]{Radiative Capture}

    Placeholder

\end{frame}

\section{Recoil Separation}

\begin{frame}[fragile]{Particle Selection}
    We can uniquely identify particles by their mass, charge and energy:

    \begin{alertblock}{Magnetic Selection}
        \[
            \frac{m}{q} = \frac{B\rho}{2}\left(\frac{T}{q}\right)^{-1}
        \]
    \end{alertblock}
    \begin{alertblock}{Electric Selection}
        \[
            \frac{T}{q} = \frac{E\rho}{2}
        \]
    \end{alertblock}
    \begin{alertblock}{Wien Filter Selection}
        \[
            \frac{m}{q} = \frac{2}{v^2} \frac{T}{q}
        \]
    \end{alertblock}

    $B\rho$ and $E\rho$ are the magnetic and electric rigidities of the
    particle in question

\end{frame}

\begin{frame}[fragile]{Particle Selection}

    Any two of the three possibilities may be combined to uniquely
    identify a particle

    \begin{figure}
        \includegraphics[width=0.75\textwidth]%
            {figures/particle_selection_by_field.png}
    \end{figure}

\end{frame}

\begin{frame}[fragile]{St.\ George}

    \begin{figure}
        \includegraphics[width=0.95\textwidth]%
            {figures/stg.png}
    \end{figure}

\end{frame}

\begin{frame}[fragile]{Angular and Energy Acceptance}

    Recoils can only be transported within defined parameter bounds
    \[
        \begin{split}
            \Delta E/E = 7.5\,\%
        \end{split}
        \quad\quad
        \begin{split}
            \Delta\theta = 40\,\text{mrad}
        \end{split}
    \]

    These bounds must hold for all possible $E\rho$ and $B\rho$

\end{frame}

\begin{frame}[fragile]{Magnetic and Electric Rigidities}

    Elements within St.\ George are tuned for the $B\rho$ and $E\rho$ of
    the recoil particle

\end{frame}

\section{Commissioning}

\begin{frame}[fragile]{Types of Commissioning}

    Can divide commissioning between three possible cases:
    \begin{itemize}
        \item Energy acceptance: change the particle energy without
            interfering with the other quantities
        \item Angular acceptance: change the deflection of the particle
            at the target location
        \item Joint acceptance: adjust both at the same time
    \end{itemize}

\end{frame}

\begin{frame}[fragile]{Energy Acceptance}

    Show Meisel et al. figure (or adaptation)

\end{frame}

\begin{frame}[fragile]{Angular Acceptance}

    Discuss deflector chamber

\end{frame}

\begin{frame}[fragile]{Joint Acceptance}

    Discuss degrader foil

\end{frame}

\section{Reaction Measurements}

\begin{frame}[fragile]{Justification for Reaction Measurements}

    Placeholder

\end{frame}

\begin{frame}[fragile]{The \alpa{} Reaction}

    Placeholder

\end{frame}

\begin{frame}[fragile]{Alternate Tune}

    Since the last segment of St.\ George has not been fully
    commissioned (full angular acceptance not yet verified), we can use
    the focal plane after the Wien filter to perform cross section
    measurements

    For $({\rm p},\alpha)$ reactions, the beam suppression is sufficient
    to reject the high-intensity incident beam

    The beam spot at this focal plane needs to be adjusted to direct the
    reaction products to the detector

\end{frame}

\begin{frame}[fragile]{Alternate Tune}

    Show COSY raytrace

\end{frame}

\begin{frame}[fragile]{Angular Acceptance Measurement}

    Since \alpa{} emits the $\alpha$ particles within a large cone, we
    can measure the acceptance of those that pass into St.\ George
    \begin{itemize}
        \item Target cup restricts us to $\approx 40$~mrad
    \end{itemize}

    Tuning the beginning of the separator to transport the particles to
    the Wien filter focal plane is an alternative to a full angular
    acceptance measurement

    Ability to fine-tune and have confidence in the properties of the
    separator are required

\end{frame}

\begin{frame}[fragile]{Acceptance Measurements}

    Show acceptance bands plot

\end{frame}

\begin{frame}[fragile]{Acceptance Measurements}

    On-resonance, we are consistent with a $\Delta\theta = 40$~mrad
    acceptance

    Off-resonance, we have larger uncertainty and are limited by a lack
    of diagnostics

    Systematic trends around each resonance can be due to non-uniform
    angular acceptance (seen during preliminary tests)

\end{frame}

\section{Future Directions}

\begin{frame}[fragile]{Future Directions}

    St.\ George can be used to study an additional class of reactions

    Solid and gas targets may be used for experiments

    Separation properties and beam currents are suitable for low-energy
    and off-resonance cross section measurements

\end{frame}

\begin{frame}[fragile]{Lessons Learned}

    Placeholder

\end{frame}

\appendix

\begin{frame}[fragile]{Backup slides}

    Placeholder

\end{frame}

% \begin{frame}[allowframebreaks]{References}
%   \bibliography{defense}
%   \bibliographystyle{abbrv}
% \end{frame}

\end{document}
