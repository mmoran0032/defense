\documentclass[10pt]{beamer}

\usetheme[progressbar=frametitle, titleformat=smallcaps]{metropolis}
\usepackage{appendixnumberbeamer}

\usepackage{booktabs}
\usepackage[scale=2]{ccicons}

\usepackage{pgfplots}
\usepgfplotslibrary{dateplot}

\usepackage{xspace}
\newcommand{\themename}{\textbf{\textsc{metropolis}}\xspace}

\newcommand{\nuc}[2]{${}^{#1}\textrm{#2}$}
\newcommand{\mnuc}[2]{{}^{#1}\textrm{#2}}
\newcommand{\react}[4]{$#1(#2,#3)#4$}
\newcommand{\mreact}[4]{#1(#2,#3)#4}
\newcommand{\alpa}{\react{\mnuc{27}{Al}}{\textrm{p}}{\alpha}{\mnuc{24}{Mg}}}
\newcommand{\nag}{\react{\mnuc{14}{N}}{\alpha}{\gamma}{\mnuc{18}{F}}}
\newcommand{\squared}{${}^{2}$}

\title{
    Verification of Recoil Separator Properties Through Reaction Measurements
}
\subtitle{}
% \date{\today}
\date{November 9, 2018}
\author{Michael T Moran}
\institute{University of Notre Dame}
% \titlegraphic{\hfill\includegraphics[height=1.5cm]{logo.pdf}}

\begin{document}

\maketitle

% \begin{frame}[fragile]{Table of Contents}
%   \setbeamertemplate{section in toc}[sections numbered]
%   \tableofcontents[hideallsubsections]
% \end{frame}

\section{Introduction}

\begin{frame}[fragile]{Prominent Reaction Processes}

    Stellar energy production depends on the type of star and the
    current lifecycle stage

    $pp$-Chains and CNO cycles dominate for stars similar to our sun
    \begin{itemize}
        \item Include radiative capture reactions: $({\rm p},\gamma)$
            and $(\alpha,\gamma)$
    \end{itemize}

    Can be grouped under ``Hydrogen burning'' processes

\end{frame}

\begin{frame}[fragile]{Hydrogen Burning}

    \begin{figure}
        \includegraphics[width=0.95\textwidth]%
            {figures/cno_nena_mgal.png}
    \end{figure}

\end{frame}

\begin{frame}[fragile]{Radiative Capture}

    Reactions of the form \react{A}{a}{\gamma}{B}

    Commonly studied by detecting the emitted $\gamma$
    \begin{itemize}
        \item Can have large background count rate, making detection
            difficult or impossible
        \item Limited by detector efficiency
    \end{itemize}

    Focus primarily on resonances

\end{frame}

\begin{frame}[fragile]{Inverse Kinematics}

    We can instead detect the heavy recoil particle using a
    high-efficiency detector to study complete cross sections
    \begin{itemize}
        \item Perform the reaction in inverse kinematics
            \react{a}{A}{B}{\gamma}
        \item Heavy projectile impinges on light target, heavy recoil
            escapes the target
        \item Requires a stable H or He (commonly) target
    \end{itemize}

    Gain in efficiency of detector offset by complexities of
    distinguishing the few recoil particles from the high-intensity beam

\end{frame}

\section{Recoil Separation}

\begin{frame}[fragile]{St.\ George}

    \begin{figure}
        \includegraphics[width=0.95\textwidth]%
            {figures/stg.png}
    \end{figure}

\end{frame}

\begin{frame}[fragile]{Magnetic and Electric Rigidities}

    Elements within St.\ George are tuned for the $B\rho$ and $E\rho$ of
    the recoil particle
    \[
        \begin{split}
            B\rho = \frac{\sqrt{2mT}}{q}
        \end{split}
        \quad\quad
        \begin{split}
            E\rho = \frac{2T}{q}
        \end{split}
    \]

    Design limits: $0.1 \leq B\rho \leq 0.45$~Tm and $E\rho \leq 5.7$~MV

\end{frame}

\begin{frame}[fragile]{Particle Selection}

    We can uniquely identify particles by their mass, charge and energy:

    \begin{alertblock}{Magnetic Selection}
        \[
            \frac{m}{q} = \frac{B\rho}{2}\left(\frac{T}{q}\right)^{-1}
        \]
    \end{alertblock}
    \begin{alertblock}{Electric Selection}
        \[
            \frac{T}{q} = \frac{E\rho}{2}
        \]
    \end{alertblock}
    \begin{alertblock}{Wien Filter Selection}
        \[
            \frac{m}{q} = \frac{2}{v^2} \frac{T}{q}
        \]
    \end{alertblock}

    % $B\rho$ and $E\rho$ are the magnetic and electric rigidities of the
    % particle in question

\end{frame}

\begin{frame}[fragile]{Particle Selection}

    Any two of the three possibilities may be combined to uniquely
    identify a particle

    \begin{figure}
        \includegraphics[width=0.75\textwidth]%
            {figures/particle_selection_by_field.png}
    \end{figure}

\end{frame}

\begin{frame}[fragile]{Angular and Energy Acceptance}

    Recoils can only be transported within defined parameter bounds
    \[
        \begin{split}
            \Delta E/E = 7.5\,\%
        \end{split}
        \quad\quad
        \begin{split}
            \Delta\theta = 40\,\text{mrad}
        \end{split}
    \]

    These bounds must hold for all possible $E\rho$ and $B\rho$

\end{frame}

\begin{frame}[fragile]{Importance of Acceptances}

    Verifying the acceptances across a wide range of $B\rho$ and $E\rho$
    is required in order to eliminate it as an unknown source of error

    Ensures that all of the produced recoils for a given reaction reach
    the detector plane
    \begin{itemize}
        \item The produced recoils can be extremely rare ($10^{-15}$ per
            beam particle)
    \end{itemize}

    Once acceptances have been verified for enough $B\rho$ and $E\rho$
    possibilities, scaling the electromagnetic elements to other
    rigidity values should retain the acceptance properties

\end{frame}

\section{Commissioning}

\begin{frame}[fragile]{Types of Commissioning}

    Can divide commissioning between three possible cases:

    \begin{alertblock}{Energy}<1->%<1-| alert@4> for highlighting
        change the particle energy without interfering with the other
        quantities
    \end{alertblock}
    \begin{alertblock}{Angular}<2->
        change the deflection of the particle at the target location
    \end{alertblock}
    \begin{alertblock}{Joint}<3->
        adjust both at the same time
    \end{alertblock}

    The goal is to get 100\,\% of the recoil particles to the final
    detector plane

\end{frame}

% \begin{frame}[fragile]{Goal of Commissioning}

%     For a properly-tuned separator, 100\,\% of the recoil particles will
%     be transmitted to the final detector plane
%     \begin{itemize}
%         \item Can measure current at the target and detector location
%         \item Can use a particle detector (if current low enough)
%     \end{itemize}

%     Will be true for particles within the designed angular and energy
%     acceptance windows for a single tune of the separator

% \end{frame}

% \begin{frame}[fragile]{Energy Acceptance}

%     For a particle beam with a given $B\rho$ and $E\rho$:

%     \begin{itemize}
%         \item<1-| alert@1> Tune the test beam to a given energy, and
%             tune St.\ George for that energy
%         \item<2-| alert@2> Verify 100\,\% transmission, adjust the tune
%             if necessary
%         \item<3-| alert@3> Adjust the beam energy within the energy
%             acceptance bounds and measure transmission
%         \item<4-| alert@4> Adjust the tune if necessary to have 100\,\%
%             transmission for all possible energy changes within the
%             acceptance bounds
%     \end{itemize}

% \end{frame}

\begin{frame}[fragile]{Energy Acceptance}

    \begin{figure}
        \includegraphics[width=0.75\textwidth]%
            {figures/rigidity_phase_space.png}
    \end{figure}

\end{frame}

% \begin{frame}[fragile]{Angular Acceptance}

%     For a particle beam with a given $B\rho$ and $E\rho$:

%     \begin{itemize}
%         \item<1-| alert@1> Tune the test beam to a given energy, and
%             tune St.\ George for that energy
%         \item<2-| alert@2> Verify 100\,\% transmission, adjust the tune
%             if necessary
%         \item<3-| alert@3> Deflect the beam at the target location
%             within the acceptance bounds (horizontally and vertically)
%         \item<4-| alert@4> Adjust the tune if necessary to have 100\,\%
%             transmission for all possible angular changes within the
%             acceptance bounds
%     \end{itemize}

% \end{frame}

\begin{frame}[fragile]{Joint Acceptance}

    All experiments will have an angular and energy spread, so must
    confirm that the acceptances can be achieved at the same time
    % \begin{itemize}
    %     \item Following the previous procedures, but interwoven, would
    %         take too long for each point
    % \end{itemize}

    Can use a degrader foil to create an angular and energy spread at
    the same time
    \begin{itemize}
        \item New central energy based on energy loss
        \item Target material and thickness extremely important to
            understand well
        \item ``Fuzziness'' of beam spot may still make it difficult
            to tune
    \end{itemize}

\end{frame}

\section{Reaction Measurements}

\begin{frame}[fragile]{Justification for Reaction Measurements}

    Joint acceptance measurements are costly to cover all of the
    possibilities

    Reactions can be used to ``bootstrap'' the process
    \begin{itemize}
        \item Reactions have a known cross section, angular and energy
            spread, etc.
        \item If all of the expected produced recoils reach the
            detector, the separator is performing optimally
    \end{itemize}

\end{frame}

\begin{frame}[fragile]{The \alpa{} Reaction}

    \begin{figure}
        \includegraphics[width=0.75\textwidth]%
            {figures/reaction_energy_diagram.png}
    \end{figure}

\end{frame}

\begin{frame}[fragile]{The \alpa{} Reaction}

    \begin{figure}
        \includegraphics[width=0.75\textwidth]%
            {figures/zero_degree_pa0.png}
    \end{figure}

\end{frame}

\begin{frame}[fragile]{Alternate Tune}

    Since the last segment of St.\ George has not been fully
    commissioned (full angular acceptance not yet verified), we can use
    the focal plane after the Wien filter to perform cross section
    measurements

    For $({\rm p},\alpha)$ reactions, the beam suppression is sufficient
    to reject the high-intensity incident beam

    The beam spot at this focal plane needs to be adjusted to direct the
    reaction products to the detector

\end{frame}

\begin{frame}[fragile]{Alternate Tune}

    \begin{figure}
        \includegraphics[width=0.75\textwidth]%
            {figures/optimal_tune_x.png}<1>
        \includegraphics[width=0.75\textwidth]%
            {figures/optimal_tune_y.png}<2>
    \end{figure}

\end{frame}

\begin{frame}[fragile]{Angular Acceptance Measurement}

    Since \alpa{} emits the $\alpha$ particles within a large cone, we
    can measure the acceptance of those that pass into St.\ George
    \begin{itemize}
        \item Target cup restricts us to $\approx 40$~mrad
    \end{itemize}

    Tuning the beginning of the separator to transport the particles to
    the Wien filter focal plane is an alternative to a full angular
    acceptance measurement

    Ability to fine-tune and have confidence in the properties of the
    separator are required

\end{frame}

\begin{frame}[fragile]{Acceptance Measurements}

    \begin{figure}
        \includegraphics[width=0.75\textwidth]%
            {figures/acceptance_uncertainty.png}
    \end{figure}

\end{frame}

% \begin{frame}[fragile]{Acceptance Measurements}

%     On-resonance, we are consistent with a $\Delta\theta = 40$~mrad
%     acceptance

%     Off-resonance, we have larger uncertainty and are limited by a lack
%     of diagnostics

%     Systematic trends around each resonance can be due to non-uniform
%     angular acceptance (seen during preliminary tests)

% \end{frame}

\section{Future Directions}

\begin{frame}[fragile]{Experimental Outcome}

    Acceptances:
    \begin{itemize}
        \item $\Delta E/E \geq 7.5$\,\%
        \item $\Delta\theta = 40$~mrad (to WF)
        \item $\Delta E/E = 3$\,\% and $\Delta\theta = 40$~mrad (to WF)
    \end{itemize}

    St.\ George can be used to study an additional class of reactions

    Separation properties and beam currents are suitable for low-energy
    and off-resonance cross section measurements

\end{frame}

\begin{frame}[fragile]{The Future of St. George}

    Final commissioning work for the remainder of St.\ George

    New supersonic jet gas target HIPPO to be commissioned
    \begin{itemize}
        \item Includes offset Si detector for continuous current
            measurements
    \end{itemize}

    First $(\alpha,\gamma)$ reactions performed

\end{frame}

\maketitle

\appendix

% potential slides:
% - tables showing the uncertainty values
% - detailed run plan
% - astrophysics support? PP, CNO, etc?
% - R-matrix (plus extra details)
% - STG planned reactions
% - detector spectra

\begin{frame}[fragile]{95\,\% Contributions}  % build out backup slides

    \begin{figure}
        \includegraphics[width=0.7\textwidth]%
            {figures/table_95.png}
    \end{figure}

\end{frame}

\begin{frame}[fragile]{Initial Test Reactions}

    \begin{figure}
        \includegraphics[width=0.95\textwidth]%
            {figures/stg_reactions.png}
    \end{figure}

\end{frame}

% \begin{frame}[allowframebreaks]{References}
%   \bibliography{defense}
%   \bibliographystyle{abbrv}
% \end{frame}

\end{document}
